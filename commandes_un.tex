%%%%%%%%%%     NEW COMMANDS

\newcommand{\RR}{\mathbb{R}}
\newcommand{\ZZ}{\mathbb{Z}}
\newcommand{\ZP}{\mathbb{Z}_{>0}}
\newcommand{\K}{\mathrm{K}}
\newcommand{\J}{\mathrm{J}}
\newcommand{\La}{\mathcal{L}}
\newcommand{\Tam}[1]{\mathrm{PBT}_{#1}}
\newcommand{\PT}[1]{\mathrm{PT}_{#1}}
\newcommand{\CT}[1]{\mathrm{CT}_{#1}}
\newcommand{\CMT}[1]{\mathrm{CMT}_{#1}}
\newcommand{\sC}{\mathcal{C}}
\newcommand{\PolySub}{\mathsf{Poly}}
\renewcommand{\Top}{\mathsf{Top}}
\newcommand{\CW}{\mathsf{CW}}
\newcommand{\sF}{\mathcal{F}}
\DeclareMathOperator{\tp}{top}
\DeclareMathOperator{\bm}{bot}
\DeclareMathOperator{\pr}{pr}
\DeclareMathOperator{\conv}{conv}
\DeclareMathOperator{\diam}{diam}
\newcommand{\ang}[1]{\langle #1\rangle}
\newcommand{\No}{\mathcal{N}}
\newcommand{\Ve}{\mathcal{V}}
\newcommand{\Bool}[1]{\{0, 1\}^{#1}}
\newcommand{\sk}{\mathrm{sk}}
\DeclareMathOperator{\vol}{vol}
\newcommand{\tr}{\mathrm{tr}}
\newcommand{\id}{\mathrm{id}}
\newcommand{\itd}[1]{\triangle^{(#1)}}
\DeclareMathOperator*{\colim}{\mathrm{colim}}
\DeclareMathOperator{\im}{\mathrm{Im}}
\newcommand{\abs}[1]{\lvert #1\rvert}
\DeclareMathOperator{\codim}{codim}
\newcommand{\B}{\mathrm{B}}
\newcommand{\T}{\mathrm{T}}
\newcommand{\iBT}{\mathrm{i}^\T_\B}
\newcommand{\cat}[1]{\mathcal{#1}}
\newcommand{\gra}{\ell}
\newcommand{\Ainf}{\ensuremath{\mathrm{A}_\infty}}
\newcommand{\Ainfdeux}{\ensuremath{\mathrm{A}_\infty^2}}
\newcommand{\uAinf}{\ensuremath{\mathrm{uA}_\infty}}%A-infinity
\newcommand{\AAinf}{\mathrm{AA}_\infty} %AA-infinity
\newcommand{\Minf}{\ensuremath{\mathrm{M}_\infty}} %A-infinity bimodule
\newcommand{\Aalg}{\ensuremath{\mathrm{A}_\infty\text{-\hspace{2pt}}\mathsf{alg}}} %category A-infinity
\newcommand{\infAalg}{\ensuremath{\infty\text{-}\mathrm{A}_\infty\text{-}\mathsf{alg}}} %category A-infinity with morphisms
\newcommand{\ide}{\ensuremath{\mathrm{id}}}
\newcommand{\mono}{\ensuremath{\mathrm{mono}}}
\newcommand{\comp}{\ensuremath{\mathrm{comp}}}
\newcommand{\ombas}{\ensuremath{\Omega B As}}
\newcommand{\diagainf}{\ensuremath{\triangle^{K}}}
\newcommand{\diagminf}{\ensuremath{\triangle^{J}}}

%OEIS
\newcommand{\OEIS}[1]{{\rm \href{http://oeis.org/#1}{\texttt{#1}}}}

%Mots parentheses colores
\newcommand{\black}[1]{\boldsymbol{\left(\right.} #1 \boldsymbol{\left.\right)}}
\newcommand{\blue}[1]{\textcolor{MidnightBlue}{\boldsymbol{\left(\right.}} #1 \textcolor{MidnightBlue}{\boldsymbol{\left.\right)}}} 
\newcommand{\red}[1]{\textcolor{BrickRed}{\boldsymbol{\left(\right.}} #1 \textcolor{BrickRed}{\boldsymbol{\left.\right)}}} 
\newcommand{\purple}[1]{\textcolor{Purple}{\boldsymbol{\left(\right.}} #1 \textcolor{Purple}{\boldsymbol{\left.\right)}}} 
\newcommand{\yellow}[1]{\textcolor{Dandelion}{\boldsymbol{\left(\right.}} #1 \textcolor{Dandelion}{\boldsymbol{\left.\right)}}} 
\newcommand{\green}[1]{\textcolor{ForestGreen}{\boldsymbol{\left(\right.}} #1 \textcolor{ForestGreen}{\boldsymbol{\left.\right)}}}
\newcommand{\orange}[1]{\textcolor{BurntOrange}{\boldsymbol{\left(\right.}} #1 \textcolor{BurntOrange}{\boldsymbol{\left.\right)}}} 

%textes colorés
\newcommand{\rouge}[1]{\textcolor{red}{#1}}

%%%% Trees NEW

\newcommand{\pointbullet}{
\begin{tikzpicture}[very thick]
\draw[fill,black] (0,0) circle (0.5) ;
\end{tikzpicture}}

\newcommand{\TreeLa}
{\vcenter{\hbox{
\begin{tikzpicture}[scale=0.25]
\draw[very thick, MidnightBlue] (0,-0.5)--(0,0) -- (-2,2)--(-2,2.5);
\draw[very thick, MidnightBlue] (-1,1)--(0,2)--(0,2.5) ;
\draw[very thick, MidnightBlue] (0,0)--(1,1)--(1,2.5) ;
%
\draw[very thick, BrickRed] (0,-0.5)--(0,-1) ;
%
\draw (-0.5,-0.5) --(0.5, -0.5);
\end{tikzpicture}}}}

\newcommand{\TreeLb}
{\vcenter{\hbox{
\begin{tikzpicture}[scale=0.25]
\draw[very thick, MidnightBlue] (-0.5,0.5)--(-2,2)--(-2,2.5);
\draw[very thick, MidnightBlue] (-1,1)--(0,2)--(0,2.5) ;
\draw[very thick, MidnightBlue] (0.5,0.5)--(1,1)--(1,2.5) ;
%
\draw[very thick, BrickRed] (0,0)--(0,-0.5) ;
\draw[very thick, BrickRed] (0,0)--(-0.5, 0.5) ;
\draw[very thick, BrickRed] (0,0)--(0.5,0.5) ;
%
\draw (-1,0.5) --(1, 0.5);
\end{tikzpicture}}}}

\newcommand{\TreeLc}
{\vcenter{\hbox{
\begin{tikzpicture}[scale=0.25]
\draw[very thick, MidnightBlue] (-2, 2.5)--(-2, 3) ;
\draw[very thick, MidnightBlue] (0, 2.5)--(0, 3) ;
\draw[very thick, MidnightBlue] (1, 2.5)--(1, 3) ;
%
\draw[very thick, BrickRed] (0,-0.5)--(0,0) -- (-2,2)--(-2,2.5);
\draw[very thick, BrickRed] (-1,1)--(0,2)--(0,2.5) ;
\draw[very thick, BrickRed] (0,0)--(1,1)--(1,2.5) ;
%
\draw (-2.5,2.5) --(1.5, 2.5);
\end{tikzpicture}}}}

\newcommand{\TreeRa}
{\vcenter{\hbox{
\begin{tikzpicture}[scale=0.25]
\draw[very thick, MidnightBlue] (0,-0.5)--(0,0) -- (2,2)--(2,2.5);
\draw[very thick, MidnightBlue] (1,1)--(0,2)--(0,2.5) ;
\draw[very thick, MidnightBlue] (0,0)--(-1,1)--(-1,2.5) ;
%
\draw[very thick, BrickRed] (0,-0.5)--(0,-1) ;
%
\draw (-0.5,-0.5) --(0.5, -0.5);
\end{tikzpicture}}}}

\newcommand{\TreeRb}
{\vcenter{\hbox{
\begin{tikzpicture}[scale=0.25]
\draw[very thick, MidnightBlue] (0.5,0.5) -- (2,2)--(2,2.5);
\draw[very thick, MidnightBlue] (1,1)--(0,2)--(0,2.5) ;
\draw[very thick, MidnightBlue] (-0.5,0.5)--(-1,1)--(-1,2.5) ;
%
\draw[very thick, BrickRed] (0,-0.5)--(0,0) ;
\draw[very thick, BrickRed] (0,0)--(0.5,0.5) ;
\draw[very thick, BrickRed] (0,0)--(-0.5,0.5) ;
%
\draw (-1,0.5) --(1, 0.5);
\end{tikzpicture}}}}

\newcommand{\TreeRc}
{\vcenter{\hbox{
\begin{tikzpicture}[scale=0.25]
\draw[very thick, BrickRed] (0,-0.5)--(0,0) -- (2,2)--(2,2.5);
\draw[very thick, BrickRed] (1,1)--(0,2)--(0,2.5) ;
\draw[very thick, BrickRed] (0,0)--(-1,1)--(-1,2.5) ;
%
\draw[very thick, MidnightBlue] (2,2.5)--(2,3) ;
\draw[very thick, MidnightBlue] (0,2.5)--(0,3) ;
\draw[very thick, MidnightBlue] (-1,2.5)--(-1,3) ;
%
\draw (-1.5,2.5) --(2.5, 2.5);
\end{tikzpicture}}}}

\newcommand{\TreeLab}
{\vcenter{\hbox{
\begin{tikzpicture}[scale=0.25]
\draw[very thick, MidnightBlue] (0,0) -- (-2,2)--(-2,2.5);
\draw[very thick, MidnightBlue] (-1,1)--(0,2)--(0,2.5) ;
\draw[very thick, MidnightBlue] (0,0)--(1,1)--(1,2.5) ;
%
\draw[very thick, BrickRed] (0,0)--(0,-0.5) ;
%
\draw (-0.5,0) --(0.5, 0);
\end{tikzpicture}}}}

\newcommand{\TreeLbc}
{\vcenter{\hbox{
\begin{tikzpicture}[scale=0.25]
\draw[very thick, MidnightBlue] (-1,1) -- (-2,2)--(-2,2.5);
\draw[very thick, MidnightBlue] (-1,1)--(0,2)--(0,2.5) ;
\draw[very thick, MidnightBlue] (1,1)--(1,2.5) ;
%
\draw[very thick, BrickRed] (0,-0.5)--(0,0) ;
\draw[very thick, BrickRed] (0,0)--(-1,1) ;
\draw[very thick, BrickRed] (0,0)--(1,1) ;
%
\draw (-1.5,1) --(1.5, 1);
\end{tikzpicture}}}}

\newcommand{\TreeRab}
{\vcenter{\hbox{
\begin{tikzpicture}[scale=0.25]
\draw[very thick, MidnightBlue] (0,0) -- (2,2)--(2,2.5);
\draw[very thick, MidnightBlue] (1,1)--(0,2)--(0,2.5) ;
\draw[very thick, MidnightBlue] (0,0)--(-1,1)--(-1,2.5) ;
%
\draw[very thick, BrickRed] (0,-0.5)--(0,0) ;
%
\draw (-0.5,0) --(0.5, 0);
\end{tikzpicture}}}}

\newcommand{\TreeRbc}
{\vcenter{\hbox{
\begin{tikzpicture}[scale=0.25]
\draw[very thick, MidnightBlue] (1,1) -- (2,2)--(2,2.5);
\draw[very thick, MidnightBlue] (1,1)--(0,2)--(0,2.5) ;
\draw[very thick, MidnightBlue] (-1,1)--(-1,2.5) ;
%
\draw[very thick, BrickRed] (0,-0.5)--(0,0) ;
\draw[very thick, BrickRed] (0,0)--(1,1) ;
\draw[very thick, BrickRed] (0,0)--(-1,1) ;
%
\draw (-1.5,1) --(1.5, 1);
\end{tikzpicture}}}}

\newcommand{\TreeCa}
{\vcenter{\hbox{
\begin{tikzpicture}[scale=0.25]
\draw[very thick, MidnightBlue] (0,-0.5) -- (0,1.5);
\draw[very thick, MidnightBlue] (0,0) -- (-1,1)--(-1,1.5);
\draw[very thick, MidnightBlue] (0,0) -- (1,1)--(1,1.5);
%
\draw[very thick, BrickRed] (0,-0.5)--(0,-1) ;
%
\draw (-0.5,-0.5) --(0.5, -0.5);
\end{tikzpicture}}}}

\newcommand{\TreeCb}
{\vcenter{\hbox{
\begin{tikzpicture}[scale=0.25]
\draw[very thick, MidnightBlue] (0,-0.5) -- (0,1.5);
\draw[very thick, MidnightBlue] (0,0) -- (-1,1)--(-1,1.5);
\draw[very thick, MidnightBlue] (0,0) -- (1,1)--(1,1.5);
%
\draw[very thick, BrickRed] (-1,1.5)--(-1,2) ;
\draw[very thick, BrickRed] (0,1.5)--(0,2) ;
\draw[very thick, BrickRed] (1,1.5)--(1,2) ;
%
\draw (-1.5,1.5) --(1.5, 1.5);
\end{tikzpicture}}}}

\newcommand{\TreeCab}
{\vcenter{\hbox{
\begin{tikzpicture}[scale=0.25]
\draw[very thick, MidnightBlue] (0,0) -- (0,1.5);
\draw[very thick, MidnightBlue] (0,0) -- (-1,1)--(-1,1.5);
\draw[very thick, MidnightBlue] (0,0) -- (1,1)--(1,1.5);
%
\draw[very thick, BrickRed] (0,-0.5)--(0,0) ;
%
\draw (-0.5,0) --(0.5, 0);
\end{tikzpicture}}}}

\newcommand{\TreeIab}
{\vcenter{\hbox{
\begin{tikzpicture}[scale=0.35]
\draw[very thick, MidnightBlue] (0,0.5)--(0,0);
%
\draw[very thick, BrickRed] (0,-0.5)--(0,0) ;
%
\draw (-0.25,0) --(0.25, 0);
\end{tikzpicture}}}}

\newcommand{\TreeBa}
{\vcenter{\hbox{
\begin{tikzpicture}[scale=0.25]
\draw[very thick, MidnightBlue] (0,0.5)--(0,1);
\draw[very thick, MidnightBlue] (0,1)--(-0.5,1.5)--(-0.5,2);
\draw[very thick, MidnightBlue] (0,1)--(0.5,1.5)--(0.5,2);
% 
\draw[very thick, BrickRed] (0,0)--(0,0.5);
\draw (-0.5,0.5) --(0.5, 0.5);\end{tikzpicture}}}}

\newcommand{\TreeBb}
{\vcenter{\hbox{
\begin{tikzpicture}[scale=0.25]
\draw[very thick, MidnightBlue] (-0.5,2)--(-0.5,2.5);
\draw[very thick, MidnightBlue] (0.5,2)--(0.5,2.5);
%
\draw[very thick, BrickRed] (0,0.5)--(0,1);
\draw[very thick, BrickRed] (0,1)--(-0.5,1.5)--(-0.5,2);
\draw[very thick, BrickRed] (0,1)--(0.5,1.5)--(0.5,2);
% 
\draw (-1,2) --(1, 2);
\end{tikzpicture}}}}

\newcommand{\TreeBab}
{\vcenter{\hbox{
\begin{tikzpicture}[scale=0.25]
\draw[very thick, MidnightBlue] (0,1)--(-0.5,1.5)--(-0.5,2);
\draw[very thick, MidnightBlue] (0,1)--(0.5,1.5)--(0.5,2);
% 
\draw[very thick, BrickRed] (0,0.5)--(0,1);
\draw (-0.5,1) --(0.5,1);
\end{tikzpicture}}}}

%%%% Trees OLD

\newcommand{\TreeR}
{\vcenter{\hbox{
\begin{tikzpicture}[yscale=0.2,xscale=0.2]
\draw[thick] (0,-1)--(0,0) -- (2,2);
\draw[thick] (1,1)--(0,2) ;
\draw[thick] (0,0)--(-1,1) ;
\draw [fill] (0,-1) circle [radius=0.035];
\draw [fill] (2,2) circle [radius=0.035];
\draw [fill] (0,2) circle [radius=0.035];
\draw [fill] (-1,1) circle [radius=0.035];
\end{tikzpicture}}}}
\newcommand{\TreeL}
{\vcenter{\hbox{
\begin{tikzpicture}[yscale=0.2,xscale=0.2]
\draw[thick] (0,-1)--(0,0) -- (-2,2);
\draw[thick] (-1,1)--(0,2) ;
\draw[thick] (0,0)--(1,1) ;
\draw [fill] (0,-1) circle [radius=0.035];
\draw [fill] (-2,2) circle [radius=0.035];
\draw [fill] (0,2) circle [radius=0.035];
\draw [fill] (1,1) circle [radius=0.035];
\end{tikzpicture}}}}
\newcommand{\TreeLL}
{\vcenter{\hbox{
\begin{tikzpicture}[yscale=0.2,xscale=0.2]
\draw[thick] (0,-1)--(0,0) -- (-3,3);
\draw[thick] (-1,1)--(0,2) ;
\draw[thick] (-2,2)--(-1,3) ;
\draw[thick] (0,0)--(1,1) ;
\draw [fill] (0,-1) circle [radius=0.035];
\draw [fill] (-2,2) circle [radius=0.035];
\draw [fill] (0,2) circle [radius=0.035];
\draw [fill] (1,1) circle [radius=0.035];
\draw [fill] (-3,3) circle [radius=0.035];
\draw [fill] (-1,3) circle [radius=0.035];
\end{tikzpicture}}}}
\newcommand{\TreeLR}
{\vcenter{\hbox{
\begin{tikzpicture}[yscale=0.2,xscale=0.2]
\draw[thick] (0,-1)--(0,0) -- (-2,2);
\draw[thick] (-1,1)--(1,3) ;
\draw[thick] (0,0)--(1,1) ;
\draw[thick] (0,2)--(-1,3) ;
\draw [fill] (0,-1) circle [radius=0.035];
\draw [fill] (-2,2) circle [radius=0.035];
\draw [fill] (0,2) circle [radius=0.035];
\draw [fill] (1,1) circle [radius=0.035];
\draw [fill] (-1,3) circle [radius=0.035];
\draw [fill] (1,3) circle [radius=0.035];
\end{tikzpicture}}}}
\newcommand{\TreeRR}
{\vcenter{\hbox{
\begin{tikzpicture}[yscale=0.2,xscale=0.2]
\draw[thick] (0,-1)--(0,0) -- (3,3);
\draw[thick] (1,1)--(0,2) ;
\draw[thick] (2,2)--(1,3) ;
\draw[thick] (0,0)--(-1,1) ;
\draw [fill] (0,-1) circle [radius=0.035];
\draw [fill] (2,2) circle [radius=0.035];
\draw [fill] (0,2) circle [radius=0.035];
\draw [fill] (-1,1) circle [radius=0.035];
\draw [fill] (3,3) circle [radius=0.035];
\draw [fill] (1,3) circle [radius=0.035];
\end{tikzpicture}}}}
\newcommand{\TreeRL}
{\vcenter{\hbox{
\begin{tikzpicture}[yscale=0.2,xscale=0.2]
\draw[thick] (0,-1)--(0,0) -- (2,2);
\draw[thick] (1,1)--(-1,3) ;
\draw[thick] (0,0)--(-1,1) ;
\draw[thick] (0,2)--(1,3) ;
\draw [fill] (0,-1) circle [radius=0.035];
\draw [fill] (2,2) circle [radius=0.035];
\draw [fill] (0,2) circle [radius=0.035];
\draw [fill] (-1,1) circle [radius=0.035];
\draw [fill] (1,3) circle [radius=0.035];
\draw [fill] (-1,3) circle [radius=0.035];
\end{tikzpicture}}}}
\newcommand{\TreeCC}
{\vcenter{\hbox{
\begin{tikzpicture}[yscale=0.2,xscale=0.2]
\draw[thick] (0,-1)--(0,0) -- (-2.5,2.5);
\draw[thick] (-1.5,1.5)--(-0.5,2.5) ;
\draw[thick] (1.5,1.5)--(0.5,2.5) ;
\draw[thick] (0,0)--(2.5,2.5) ;
\draw [fill] (0,-1) circle [radius=0.035];
\draw [fill] (-2.5,2.5) circle [radius=0.035];
\draw [fill] (-0.5,2.5) circle [radius=0.035];
\draw [fill] (0.5,2.5) circle [radius=0.035];
\draw [fill] (2.5,2.5) circle [radius=0.035];
\end{tikzpicture}}}}
\newcommand{\TreeAL}
{\vcenter{\hbox{
\begin{tikzpicture}[yscale=0.2,xscale=0.2]
\draw[thick] (0,-1)--(0,0) -- (-2,2);
\draw[thick] (-1,1)--(0,2) ;
\draw[thick] (0,0)--(1,1) ;
\draw[thick] (-1,1)--(-1,2) ;
\draw [fill] (0,-1) circle [radius=0.035];
\draw [fill] (-2,2) circle [radius=0.035];
\draw [fill] (0,2) circle [radius=0.035];
\draw [fill] (1,1) circle [radius=0.035];
\draw [fill] (-1,2) circle [radius=0.035];
\end{tikzpicture}}}}
\newcommand{\TreeRA}
{\vcenter{\hbox{
\begin{tikzpicture}[yscale=0.2,xscale=0.2]
\draw[thick] (0,-1)--(0,0) -- (2,2);
\draw[thick] (1,1)--(0,2) ;
\draw[thick] (0,0)--(-1,1) ;
\draw[thick] (1,1)--(1,2) ;
\draw [fill] (0,-1) circle [radius=0.035];
\draw [fill] (2,2) circle [radius=0.035];
\draw [fill] (0,2) circle [radius=0.035];
\draw [fill] (-1,1) circle [radius=0.035];
\draw [fill] (1,2) circle [radius=0.035];
\end{tikzpicture}}}}
\newcommand{\TreeAR}
{\vcenter{\hbox{
\begin{tikzpicture}[yscale=0.2,xscale=0.2]
\draw[thick] (0,-1)--(0,0) -- (2,2);
\draw[thick] (1,1)--(0,2) ;
\draw[thick] (0,0)--(-1,1) ;
\draw[thick] (0,0)--(0,1) ;
\draw [fill] (0,-1) circle [radius=0.035];
\draw [fill] (2,2) circle [radius=0.035];
\draw [fill] (0,2) circle [radius=0.035];
\draw [fill] (-1,1) circle [radius=0.035];
\draw [fill] (0,1) circle [radius=0.035];
\end{tikzpicture}}}}
\newcommand{\TreeLA}
{\vcenter{\hbox{
\begin{tikzpicture}[yscale=0.2,xscale=0.2]
\draw[thick] (0,-1)--(0,0) -- (-2,2);
\draw[thick] (-1,1)--(0,2) ;
\draw[thick] (0,0)--(1,1) ;
\draw[thick] (0,0)--(0,1) ;
\draw [fill] (0,-1) circle [radius=0.035];
\draw [fill] (-2,2) circle [radius=0.035];
\draw [fill] (0,2) circle [radius=0.035];
\draw [fill] (1,1) circle [radius=0.035];
\draw [fill] (0,1) circle [radius=0.035];
\end{tikzpicture}}}}
\newcommand{\TreeCA}
{\vcenter{\hbox{
\begin{tikzpicture}[yscale=0.2,xscale=0.2]
\draw[thick] (0,-1)--(0,0) -- (-1,1);
\draw[thick] (0,1.5)--(1,2.5) ;
\draw[thick] (0,1.5)--(-1,2.5) ;
\draw[thick] (0,0)--(1,1) ;
\draw[thick] (0,0)--(0,1.5) ;
\draw [fill] (0,-1) circle [radius=0.035];
\draw [fill] (-1,1) circle [radius=0.035];
\draw [fill] (1,2.5) circle [radius=0.035];
\draw [fill] (-1,2.5) circle [radius=0.035];
\draw [fill] (1,1) circle [radius=0.035];
\draw [fill] (0,1) circle [radius=0.035];
\end{tikzpicture}}}}
\newcommand{\TreeC}
{\vcenter{\hbox{
\begin{tikzpicture}[yscale=0.2,xscale=0.2]
\draw[thick] (0,-1.5)--(0,0);
\draw[thick] (0,0)--(1.5,1.5) ;
\draw[thick] (0,0)--(0.5,1.5) ;
\draw[thick] (0,0)--(-0.5,1.5) ;
\draw[thick] (0,0)--(-1.5,1.5) ;
\draw [fill] (0,-1.5) circle [radius=0.035];
\draw [fill] (1.5,1.5) circle [radius=0.035];
\draw [fill] (0.5,1.5) circle [radius=0.035];
\draw [fill] (-1.5,1.5) circle [radius=0.035];
\draw [fill] (-0.5,1.5) circle [radius=0.035];
\end{tikzpicture}}}}

%Drapeau européen

\usepackage{graphicx,calc}
\newlength\myheight
\newlength\mydepth
\settototalheight\myheight{Xygp}
\settodepth\mydepth{Xygp}
\setlength\fboxsep{0pt}
\newcommand*\inlinegraphics[1]{%
  \settototalheight\myheight{Xygp}%
  \settodepth\mydepth{Xygp}%
  \raisebox{-\mydepth}{\includegraphics[height=\myheight]{#1}}%
}

%%%%%%%%% Nouveau

\DeclareMathOperator{\Ima}{Im} %Image d'une fonction
\DeclareMathOperator{\cone}{Cone} %Cône

%%%%%%%%%%%   Comments 

\newcommand{\Thibaut}[1]{\textcolor{red}{#1}}
\newcommand{\Guillaume}[1]{\textcolor{blue}{#1}}